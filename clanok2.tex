\documentclass[10pt]{extarticle}
\usepackage{multicol}
\usepackage{extsizes}
\usepackage{titlesec}
\usepackage[margin={1.5cm,1.5cm}]{geometry}

\setlength\columnsep{30pt}
\titleformat*{\section}{\large\bfseries\sffamily}
\date{}
\begin{document}
\title{\LARGE \textbf{Reducing the Sparsity of Contextual Information for
Recommender Systems}}

\maketitle
\begin{multicols}{2}


\noindent\large{\textbf{ABSTRACT}} \\

\noindent \small Our  work  focuses  on  the  improvement  of  the  accuracy  of
context-aware recommender systems.  Contextual informa-
tion showed to be promising factor in recommender systems.
However, pure context-based recommender systems can not
outperform  other  approaches  mainly  due  to  high  sparsity
of contextual information.  We propose an idea to improve
accuracy of context based recommender systems by context
inference.   Context  inference  is  based  on  e ect  discovered
by analyses of the context as a factor influencing user needs.
Analyses of the news readers reveals existence of behavioural
correlation which is the main pillar of proposed context infer-
ence.  Method for context inference is based on collaborative
 ltering and clustering of web usage (as a non-discretizing
alternative to association rules mining). \\


\noindent\large\textbf{Categories and Subject Descriptors} \\


\noindent\small  H.3 [\textbf{Information Storage and Retrieval}]: Clustering In-
formation Filtering;  
H.2 [\textbf{Database Applications}]:  Datamining General Terms
Algorithms
Keywords
context, recommender system, clustering, user behaviour

\section{MOTIVATION} 

Context-aware  recommender  systems  have  become  very
popular since variety of contextual information could be ac-
quired.  With an increase of the smart-phone popularity and
available features which they provide, we are able to asso-
ciate user needs with contextual information.  From the high
level context types such as location,  time,  weather,  to the
low level context types such as humidity, noise, movement,
.
we study the impact of the context on the user behaviour and
needs.  However context itself has shown to be insuffcient
when  it  comes  to  accuracy  of  context-aware  recommender
systems.  Context is therefore used as a secondary aspect for
generating recommendation.
One  reason  for  low  accuracy  is  high  sparsity  of  contex-
tual information.  High sparsity is caused by various natures
of users and their preferences [5].  Some users do not want
to  share  their  personal  information  such  as  location,  thus
causing missing contextual information.  Poor context infor-
mation  leads  to  low  accuracy  in  prediction.   On  the  other
hand,  some users are willing to expose even personal con-
textual information such as emotions.  They are willing to
answer question and explicitly express contextual informa-
tion, which is then useful in context-aware recommendation.
Our idea is to propagate contextual information from one
user to another in order to reduce the sparsity of data.  We
propose the propagation of the context by exploiting a corre-
lation in users' behaviour.  We assume that users' behaviour
is not random,  it is based on context of the user.  For in-
stance,  Perse  [14]  discovered  association  between  negative
mood and tendency to watch competition-style programs as
a result of the need to experience happiness.   Action-style
programs are selected when viewers are in a positive emo-
tional  state.   However,  even  if  some  associations  are  valid
for majority of users, we expect that there are associations
which could be discovered only for a subset of users.  This
leads to clustering of users by their behaviour.  Identifying
clusters of similar users helps to identify how to propagate
the context between users.
We  discover  associations  between  the  need  and  context
using alternative to standard association rules mining.  The
di erence is in non-discrete values which we use.  For exam-
ple, wrong discretization causes noise, as we lose the ability
to compare them and thus sort them. Therefore we expect to
achieve higher accuracy when we use proposed value based
associations  discovery  instead  of  item  based.   To  accom-
plish  value  based  associations  discovery  we  combine  stan-
dard techniques from machine learning such as
x
-means [6]
and vector distance computation (Euclidean distance)

\section{RELATED WORK} 

Context  inference  has  received  relatively  little  attention
in the literature when it comes to implicit inference from the
logs of user activity.  It is caused by the selective approaches
to context incorporation.  Speci c solutions work with spe-
ci c contextual information.  Kahng et al. [8] demonstrate
the  prede ned  context  as  one  of  the  factors  for  document
ranking  in  information  retrieval  process.   As  an  example
they introduce weather and its impact on the user's inter-
est in song listening.  This empiric context selection emerges
from the observation made by Baltranus et al. [2].  They re-
search the relevance of the context in the system explicitly
by asking the user.  They showed that supposed context has
positive impact on the success of their method. On the other
hand, research by Asoh et al. [1] proves that there is a signif-
icant di erence between the real and supposed reaction to
the context.  One way or another, this could be understood
as an explicit form of the context acquisition.  And unfortu-
nately, we are still unable to persuade and engage everyone
into explicit feedback.
Therefore we work with the acquired context and users'
behaviour to infer missing contextual information.  To stress
the unavailability of contextual information we pick the work
of Bermingham et al. [3]. They propose a solution to discover
the sentiment from microblogs.  The sentiment is a deriva-
tion of the emotional context.  Microblogs are perfect source
for discovering this type of the context.  However it is do-
main speci c and could not be used as a generic solution.
Riboni et al. [16] announced a hybrid of statistical anal-
yses and ontological reasoning in order to acquire the con-
text.  Utilization in the COSAR project shows better results
by  combining  both  of  these  approaches.   We  have  decided
to use statistical approach boosted by empirically observed
e ect  of  users  behaviour  correlation.   We  understand  the
correlation  of  the  behaviour  as  the  correlation  of  the  con-
textual information.  Konomi et al. [10] present connections
between people formed by co-presence at places.  These con-
nections are based on geo-location but correlate with social
connections.
User  behaviour  is  often  represented  by  a  set  of  actions
performed by the user.  Kramar [11] observed the e ect of
changing the behaviour with the change of current context.
He identi ed that multiple personas are present in the be-
haviour  of  individual.   The  same  e ect  was  exploited  by
Park et al. [13].  They clustered user's behaviour by actions
to improve query suggestions. They have actually used client
side logs to cluster the behaviour, which outperformed the
state-of-art approaches.  From multiple personas of an indi-
vidual, we expanded to multiple personas of all users in the
system.  Our intention is to supplement missing contextual
information using multiple personas.
Combining multiple personas of more users will improve
results in context inference. Research made by Cadiz et al. [4]
or Rahnama et al. [15] enables us to work with more users
and in various systems.  By using standardized frameworks
and unifying context-aware systems, we are able to gather
usage logs.  Contextual information on activities from vari-
ous systems improves our abilities to infer missing context.
The only drawback of such framework is the redundancy of
some information and higher complexity. Including informa-
tion on the past,  present and future context [12] increases
the  complexity  even  more.   Several  approaches  have  been
presented to address this problem.  Komninos et al. [9] work
with  vector  representation  of  action  and  propose  solution
to reduce complexity, even the complexity caused by vector
weighting issues.  Reduction of the complexity is important
even for mobile devices where computational resources are
constrained.  Dargie et al. [7] discuss the need to reduce time
to recognise the context and its essence in real-time systems

\section{CORRELATION IN SIMILAR USERS’ BE-
HAVIOUR}

We  have  studied  the  e ect  of  correlation  in  users'  be-
haviour to propose an exploitation which would help to re-
duce the sparsity of contextual information.  Our idea is to
propagate  contextual  information  only  to  users  whose  be-
haviour highly correlates

\subsection{Contextual Information}

To prove our concept we have decided to work with database
of web usage recorded by news portal SME.sk
1
.  This news
portal  is  the  biggest  local  news  portal  with  more  than  20
thousands active readers at the peak.  Every click recorded
includes  time,  IP  address,  user  identi er  and  article  iden-
ti er.   Further  information  such  as  category,  section,  au-
thor, publishing time for article are also provided.  We used
this database before for content-based recommendation [17]
which enables us to compare results achieved by our previ-
ous work.
To prepare database for further research, we add the con-
text  which  is  not  in  database.    We  use  services  such  as
wunderground
2
and
ip2location
3
to  add  information  on
weather and location.  We also process timestamp to store
time  derivatives  (such  as  day  of  week,  part  of  day,  etc.).
Location which is extracted from IP address is very rough
and  for  dynamic  block  of  IP  addresses,  the  location  is  al-
most untraceable.  We have also applied a simple rule based
algorithm to extract information on location (home, work,
outside) using time and IP address.  It is based on repeat-
ing IP addresses during work days (from 8 AM to 5 PM),
during night and weekends.  If the IP address was used by
user during work hours many times, we add the context of
location respectively (at work).
Dataset prepared in this way contains contextual informa-
tion which is acquired with both high and low con dence.
Low con dence causes sparsity what negatively a ects fur-
ther recommendation process.

\subsection{User habits}

We have analysed news reading with focus on various con-
text types.  We presume that user has habits which are af-
fected by context.  We also presume that some users have
similar  habits  thus  their  behaviour  is  a ected  by  context
similarly.   We  have  mostly  analysed  the  time  as  the  most
popular  context  (see  Fig.   1).   The   gure  shows  that  ma-
jority  of  users  are  in uenced  by  forthcoming  events.   The
 gure proves that majority of users have similar habits.  For
instance, they read about cooking, when they are going to
cook for Christmas.
We also recognized same habits in smaller groups of users.
For  example,  local  football  games  are  commented  on  this
site,  which  attracts  some  users  with  interest  in  football.
These  users  have  same  interests.   It  could  also  mean  that
these people are also similarly in uenced by the same con-
text.   In uence  of  context  means  correlation  in  their  be-
haviour.   We  use  this  e ect  of  correlation  in  behaviour  to
form clusters of similar users.  Knowing similar users enables
us to propagate contextual information correctly

\section{USER MODEL ENRICHMENT}

Our user model represent the measure of user interest in
item.  Similarly to association rules we work with patterns.
Every pattern consist of a condition and  lter.  In our case
of  news  recommending,  the   lter  is  a  combination  of  the
section and the category in news portal.  There are around
420 combinations which could be used.
Condition expresses the context of the user which has to
be valid when the rule is applied to recommendation process.
We understand condition as a set of contexts which form a
condition together.  Conditions are used to  nd situation of
the user.  Condition and current situation of the user must
be matching when we want to apply the  lter.
User model contains only the most frequent patterns.  But
even  if  condition  does  not  match  current  situation  of  the
user, we are still able to  nd the best matching condition.
Every context has its value which represents the importance
with  a  condition.   Calculation  of  vector  distances  between
conditions and situation of the user results in best matching
rule which is then applied.

\subsection{Context Inference}

We have build the user model by processing user activity
which has been already recorded.  Some contextual informa-
tion could be acquired directly using services and processing
attributes.  However, some contextual information could be
missing or the con dence of the information is very low.  We
propose the context inference which leads to the reduction
of the sparsity.  Table (see Tab.  1) demonstrates how could
the contextual information be propagated to another.  These
actions could be represented as vectors and clustered using
all attributes except missing location.  User A has complete
contextual information in this example. User B is considered
to be similar to User A since their behaviour is similar.  In
result, contextual information for missing values is inferred
using known values and similar actions.
Context inference is basically executed in following steps

\subsection{Recommendation}

There are more options how to incorporate context into
recommender  systems.   Our  user  model  enables  us  to  rec-
ommend  items  by   ltering  them  using  rules  stored  in  the
model.  Current situation of the user is used to  nd the best
matching items.  Items are used as  lters on the dataset of
potential recommendations. Here we can work with content-
based approaches, collaborative  ltering or other.
Content-based recommendation is generated using items
which  are  fetched  from  the  model  using  current  user  con-
ditions.   In  this  alternative  we  are  searching  for  the  items
in the dataset which are similar to items from user model.
Item  in  the  user  model  is  not  necessarily  one  of  items  in
dataset.  It could be only the set of keywords.  We propose
to use category and section in our dataset of news.
Collaborative   ltering  is  another  very  popular  approach
to generate recommendations.  To use our model with this
approaches we change the pair of condition and item to con-
dition and user.  This enables us to reveal the most similar
users whose situations were very similar to the current situ-
ation of the user.

\section{EVALUATION STRATEGY}

As  we  already  mentioned  we  work  with  database  where
both low and high con dent contextual information is present.
We propose to evaluate our method for sparsity reduction by
using only records with high con dent contextual informa-
tion.  We randomly select contextual information and sim-
ulate its con dence to be very low.  Then we apply context
inference and compare inferred results with original values.
We also propose evaluation for context-aware recommen-
dation.   We  want  to  compare  results  achieved  by  recom-
mending  news  both  with  and  without  inferred  contextual
information.  Experiment is conducted with real people who
are separated into two groups.  One group receives recom-
mendations  generated  only  with  original  contextual  infor-
mation. Another group receives recommendations generated
with inferred context (reduced contextual sparsity).  In this
case we also incorporate test for statistical signi cance.
Another  approach  to  evaluate  our  approach  to  sparsity
reduction is to compare our recommender system to others
to show expected improvement

\section{CONCLUSIONS AND FUTURE WORK}

In our work we face signi cant drawback of common spar-
sity in contextual information.  We presented our proposal
for solving sparsity in contextual information using context
inference.   We  showed  analyses  of  the  web  usage  for  news
portal  and  revealed  the  e ect  of  behavioural  correlation.
Clustering  users  by  the  web  usage  splits  users  with  simi-
lar  preferences  into  groups  where  the  association  between
context  and  need  is  also  similar.    In  such  group  we  can
propagate missing contextual information or context value
which is lacking in con dence.  Our approach solves prob-
lems which are often present in frameworks which are gath-
ering contextual information from more sources.
In our future work we plan to generate news recommen-
dations  using  inferred  context  and  compare  results  to  our
previous work where we used pure content-based recommen-
dations [17].  We also plan to apply this method for smart-
phones where we encounter higher variety of context types.
Context  which  could  be  acquired  on  smart-phone  is  more
complex than on the web what is big challenge for us.
Our main contribution is in reducing the sparsity of con-
textual information thus improving accuracy of context-aware
recommender systems.  We have also designed an alternative
to association rules mining which respects numeric values.

\section{ACKNOWLEDGMENTS}

This work was partially supported by the Scienti c Grant
Agency  of  the  Ministry  of  Education  of  Slovak  Republic,
grant VG1/0675/11 and by the Slovak Research and Devel-
opment Agency under the contract No.  APVV-0208-1

\section{REFERENCES}

\end{multicols}
\end{document}