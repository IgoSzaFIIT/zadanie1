\documentclass[a4paper,8pt]{article}
\usepackage{multicol}
\usepackage{extsizes}
\usepackage[margin={1.5cm,1.5cm}]{geometry}
\setlength\columnsep{30pt}

\begin{document}
\begin{multicols}{2}


\noindent\huge\textbf{ABSTRACT} \\

\noindent Our  work  focuses  on  the  improvement  of  the  accuracy  of
context-aware recommender systems.  Contextual informa-
tion showed to be promising factor in recommender systems.
However, pure context-based recommender systems can not
outperform  other  approaches  mainly  due  to  high  sparsity
of contextual information.  We propose an idea to improve
accuracy of context based recommender systems by context
inference.   Context  inference  is  based  on  e ect  discovered
by analyses of the context as a factor influencing user needs.
Analyses of the news readers reveals existence of behavioural
correlation which is the main pillar of proposed context infer-
ence.  Method for context inference is based on collaborative
 ltering and clustering of web usage (as a non-discretizing
alternative to association rules mining).

Categories and Subject Descriptors
H.3 [
Information Storage and Retrieval
]: Clustering In-
formation Filtering;  H.2 [
Database Applications
]:  Data
mining
General Terms
Algorithms
Keywords
context, recommender system, clustering, user behaviour

\section{MOTIVATION}

Context-aware  recommender  systems  have  become  very
popular since variety of contextual information could be ac-
quired.  With an increase of the smart-phone popularity and
available features which they provide, we are able to asso-
ciate user needs with contextual information.  From the high
level context types such as location,  time,  weather,  to the
low level context types such as humidity, noise, movement,
.
we study the impact of the context on the user behaviour and
needs.  However context itself has shown to be insuffcient
when  it  comes  to  accuracy  of  context-aware  recommender
systems.  Context is therefore used as a secondary aspect for
generating recommendation.
One  reason  for  low  accuracy  is  high  sparsity  of  contex-
tual information.  High sparsity is caused by various natures
of users and their preferences [5].  Some users do not want
to  share  their  personal  information  such  as  location,  thus
causing missing contextual information.  Poor context infor-
mation  leads  to  low  accuracy  in  prediction.   On  the  other
hand,  some users are willing to expose even personal con-
textual information such as emotions.  They are willing to
answer question and explicitly express contextual informa-
tion, which is then useful in context-aware recommendation.
Our idea is to propagate contextual information from one
user to another in order to reduce the sparsity of data.  We
propose the propagation of the context by exploiting a corre-
lation in users' behaviour.  We assume that users' behaviour
is not random,  it is based on context of the user.  For in-
stance,  Perse  [14]  discovered  association  between  negative
mood and tendency to watch competition-style programs as
a result of the need to experience happiness.   Action-style
programs are selected when viewers are in a positive emo-
tional  state.   However,  even  if  some  associations  are  valid
for majority of users, we expect that there are associations
which could be discovered only for a subset of users.  This
leads to clustering of users by their behaviour.  Identifying
clusters of similar users helps to identify how to propagate
the context between users.
We  discover  associations  between  the  need  and  context
using alternative to standard association rules mining.  The
di erence is in non-discrete values which we use.  For exam-
ple, wrong discretization causes noise, as we lose the ability
to compare them and thus sort them. Therefore we expect to
achieve higher accuracy when we use proposed value based
associations  discovery  instead  of  item  based.   To  accom-
plish  value  based  associations  discovery  we  combine  stan-
dard techniques from machine learning such as
x
-means [6]
and vector distance computation (Euclidean distance)


\end{multicols}
\end{document}